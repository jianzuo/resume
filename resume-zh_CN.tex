% !TEX TS-program = xelatex
% !TEX encoding = UTF-8 Unicode
% !Mode:: "TeX:UTF-8"

\documentclass{resume}
\usepackage{zh_CN-Adobefonts_external} % Simplified Chinese Support using external fonts (./fonts/zh_CN-Adobe/)
%\usepackage{zh_CN-Adobefonts_internal} % Simplified Chinese Support using system fonts
\usepackage{linespacing_fix} % disable extra space before next section
\usepackage{cite}

\begin{document}
\pagenumbering{gobble} % suppress displaying page number

\name{左健}

% {E-mail}{mobilephone}{homepage}
% be careful of _ in emaill address
\contactInfo{(+33) 0766774816}{jianzuo0925@gmail.com}{}
{}
% {E-mail}{mobilephone}
% keep the last empty braces!
%\contactInfo{xxx@yuanbin.me}{(+86) 131-221-87xxx}{}
 
\section{个人总结}
本人热爱科研,工作认真负责,自我驱动能力强,热爱探索,学习新知识。自硕士研究生起开始接触燃料电池,氢能领域,致力于燃料电池耐久性,维护,成本优化等前沿研究。博士期间于法国继续学习研究燃料电池系统管理策略开发。 
\textbf{现为法国国家燃料电池实验室博士后(FCLAB)}。

% \section{\faGraduationCap\ 教育背景}
\section{教育背景}
\datedsubsection{\textbf{格勒诺布尔大学},自动化,\textit{博士研究生}}{2019.10 - 2022.10}
% \ \textbf{排名11/133(前10\%)},中国科学院大学学业奖学金(2次),IEEE Student member,预计2018年6月毕业
\datedsubsection{\textbf{同济大学},动力机械及工程,\textit{硕士研究生}}{2016.9 - 2019.3}
% \ \textbf{排名2/62(前5\%)},国家励志奖学金,人民奖学金(7次),科技竞赛奖(2次),北京市普通高等学校优秀毕业生,北京理工大学优秀毕业生,软件学院金牌毕业生,优秀团员/优秀学生(5次)
\datedsubsection{\textbf{上海电机学院},电气工程及其自动化,\textit{工学学士}}{2012.9 - 2016.6}
% \ 2014年中国政府奖学金(\textit{http://www.csc.edu.cn/}),DID-ACTE项目交换生(\textit{http://did-acte.org/})

% \section{\faCogs\ IT 技能}
\section{技术能力}
% increase linespacing [parsep=0.5ex]
\begin{itemize}[parsep=0.2ex]
  \item \textbf{编程语言}: Julia (熟练), Python(熟练), LaTex (熟练)
  \item 熟悉深度学习框架 Pytorch, Jax.
  % \item \textbf{操作系统与工程构建}: Linux/macOS/Git
  % \item \textbf{关键词}: React/Vue.js/D3.js(SVG)/three.js(canvas, WebGL)/chrome extension/Express
\end{itemize}

\section{项目经历}
\datedsubsection{\textbf{车用电堆及储氢瓶耐久性关键测试技术} - 上海市科学技术委员会科研计划课题}{2016.09-2019.03}
\begin{itemize}
%   \item 飞猪北京前端团队全面负责各交通线的票务(机票/火车票/汽车票) web 应用与事业群基础架构研发
  \item 课题旨在建立燃料电池辅助系统关键零部件动态性能测试及评价方法,从而解决车用电堆寿命评估挑战;
  \item 承担车用燃料电池耐久性相关实验研究及寿命衰退分析,预测;
  \item 独立完成了单片电池的长期耐久性实验 (1000h) 设计,测试 以及结果分析;建立了基于神经网络的数据驱动预测模型。
\end{itemize}
\datedsubsection{\textbf{燃料电池多电堆系统耐久性-法国国家项目 (ANR)}}{2019.10-2022.10}
\begin{itemize}
  \item 创造性的提出了基于伽玛过程的 \textbf{燃料电池随机衰退模型}。此外对于多电堆燃料电池系统衰退过程中差异性提出了基于随机因素的伽玛衰退模型。
  \item 基于提出的多电堆衰退模型,建立了随机动态工况条件下的多电堆系统功率分配策略。通过对比经典平均功率分配策略,验证了提出的策略的有效性(延长了多电堆燃料电池系统寿命)。
\end{itemize}

\datedsubsection{\textbf{燃料电池耐久性和可靠性课题} - 法国国家项目 (ANR)}{2023.01-至今}
\begin{itemize}
  \item 负责建立燃料电池系统管理决策策略 (解决当前燃料耐久性问题的关键);
  \item 整合电堆系统维护成本优化以及功率分配策略,建立基于维护的燃料电池系统管理策略 (弥补当前燃料电池技术的一个研究空白)。
\end{itemize}

% \begin{onehalfspacing}
% \end{onehalfspacing}

% \datedsubsection{\textbf{DID-ACTE} 荷兰莱顿}{2015年3月 - 2015年6月}
% \role{本科毕业设计}{LIACS 交换生}
% 利用结巴分词对中国古文进行分词与词性标注,用已有领域知识训练形成 classifier 并对结果进行调优
% \begin{onehalfspacing}
% \begin{itemize}
%   \item 利用结巴分词对中国古文进行分词与词性标注
%   \item 利用已有领域知识训练形成 classifier, 并用分词结果进行测试反馈
%   \item 尝试不同规则,对 classifier 进行调优
% \end{itemize}
% \end{onehalfspacing}

\section{学术成果}
\subsection{\textbf{期刊论文}}
% increase linespacing [parsep=0.5ex]
\begin{enumerate}[parsep=0.2ex]
%   \item LeetCodeOJ Solutions, \textit{https://github.com/hijiangtao/LeetCodeOJ}
  \item \textbf{J. Zuo}, Lv, H., Zhou, D., Xue, Q., Jin, L., Zhou, W., ... $\&$ Zhang, C. (2021). Deep learning based prognostic framework towards proton exchange membrane fuel cell for automotive application. Applied Energy, 281, 115937.  (\textbf{IF=11.446})
  \item \textbf{J. Zuo}, C. Cadet, Z. Li, C. Berenguer, R. Outbib. ``Post-prognostics decision-making strategy for load allocation on a stochastically deteriorating multi-stack fuel cell system.'' \textit{Proceedings} \textit{of} \textit{the} \textit{Institution} \textit{of} \textit{Mechanical} \textit{Engineers,} \textit{Part} \textit{O:} \textit{Journal} \textit{of} \textit{Risk} \textit{and} \textit{Reliability} (2022, \textbf{IF=1.891})
  \item \textbf{J. Zuo}, Lv, H., Zhou, D., Xue, Q., Jin, L., Zhou, W., ... $\&$ Zhang, C. (2021). Long-term dynamic durability test datasets for single proton exchange membrane fuel cell. Data in Brief, 35, 106775. (\textbf{IF=1.113})
  \item Lv, H., \textbf{J. Zuo}, Zhou, W., Shen, X., Li, B., Yang, D., ... $\&$ Zhang, C. (2019). Synthesis and activities of IrO2/Ti1− xWxO2 electrocatalyst for oxygen evolution in solid polymer electrolyte water electrolyzer. Journal of Electroanalytical Chemistry, 833, 471-479. (\textbf{IF=4.598})
  \item Wang, Chu, Manfeng Dou, Zhongliang Li, Rachid Outbib, Dongdong Zhao, \textbf{Jian Zuo}, Yuanlin Wang, Bin Liang, and Peng Wang. "Data-driven prognostics based on time-frequency analysis and symbolic recurrent neural network for fuel cells under dynamic load." Reliability Engineering \& System Safety 233 (2023): 109123. (\textbf{IF=7.247})
  \item Li, X., Zhang, Q., Chibane, H., Cavallucci, D., Tang, X., \textbf{J. Zuo}, $\&$ Song, H. (2021). Data‐Driven Temporal Charging Patterns of Electric Vehicles in China. Energy Technology, 9(12), 2100421. (\textbf{IF=4.149})
  \item Chen, G., Li, J., Lv, H., Wang, S., \textbf{J. Zuo}, $\&$ Zhu, L. (2019). Mesoporous Co x Sn (1–x) O2 as an efficient oxygen evolution catalyst support for SPE water electrolyzer. Royal Society open science, 6(4), 182223. (\textbf{IF=3.653})
  \item \textbf{J. Zuo}, C. Cadet, Z. Li, C. Berenguer, R. Outbib.“Deterioration-aware energy management strategy for a multi-stack fuel cell system under random dynamic load scenarios” (Submitted to Energy conversion and management (\textbf{IF=11.533}), \textit{under review})
\end{enumerate}
\subsection{会议论文}
\begin{enumerate}
    \item \textbf{Zuo, J.}, Cadet, C., Li, Z., Bérenguer, C., $\&$ Outbib, R. (2022, August). Fuel Cell Stochastic Deterioration Modeling for Energy Management in a Multi-stack System. In 2022 13th International Conference on Reliability, Maintainability, and Safety (ICRMS) (pp. 104-108). IEEE. (\textbf{最佳会议论文奖})
    \item \textbf{Zuo, J.}, Cadet, C., Li, Z., Bérenguer, C., $\&$ Outbib, R. (2022, August). A load allocation strategy for stochastically deteriorating multi-stack PEM fuel cells. In 32nd European Conference on Safety and Reliability (ESREL 2022) (pp. R22-01). Research publishing.
    \item \textbf{Zuo, J.}, Cadet, C., Li, Z., Bérenguer, C., $\&$ Outbib, R. (2021, May). Post-Prognostics Decision Making Strategy to Manage the Economic Lifetime of a Two-Stack PEMFC System. In 2021 Annual Reliability and Maintainability Symposium (RAMS) (pp. 1-7). IEEE.
    \item  \textbf{Zuo, J.}, Cadet, C., Li, Z., Bérenguer, C., $\&$ Outbib, R. (2020, July). Post-prognostics decision making for a two-stacks fuel cell system based on a load-dependent deterioration model. In PHME 2020-European Conference of the Prognostics and Health management (PHM) Society (Vol. 5, No. 1, p. 9).
\end{enumerate}
% \section{\faHeartO\ 项目/作品摘要}
% \section{项目/作品摘要}
% \datedline{\textit{An Integrated Version of Security Monitor Vis System}, https://hijiangtao.github.io/ss-vis-component/ }{}
% \datedline{\textit{Dark-Tech}, https://github.com/hijiangtao/dark-tech/ }{}
% \datedline{\textit{融合社交网络数据挖掘的电视节目可视化分析系统}, https://hijiangtao.github.io/variety-show-hot-spot-vis/}{}
% \datedline{\textit{LeetCodeOJ Solutions}, https://github.com/hijiangtao/LeetCodeOJ}{}
% \datedline{\textit{Info-Vis}, https://github.com/ISCAS-VIS/infovis-ucas}{}

\section{获奖情况}
\begin{itemize}
    \item 第十四届``华为杯''全国研究生数学建模竞赛国家二等奖 (硕士期间)
    \item 全国大学生数学竞赛国家三等奖 (本科期间)
    \item 全国大学生数学建模竞赛上海市二等奖 (本科期间)
\end{itemize}


% \section{\faInfo\ 社会实践/其他}
\section{学术报告}
% increase linespacing [parsep=0.5ex]
\begin{itemize}[parsep=0.2ex]
  \item Fuel cell stochastic deterioration modeling for energy management in a multi-stack system (2ème REUNION PLENIERES de la Fédération HYDROGENE (FRH2) du CNRS, Aussois, France, 2022)
  \item Fuel cell stochastic deterioration modeling for energy management in a multi-stack system (European Network for Business and Industrial Statistics, Grenoble, France, 2022)
  \item Post-prognostics decision-making strategy for load allocation on a stochastically deteriorating multi-stack fuel cell system (Gipsa-lab Safe-team meeting (Poster), Grenoble, France, 2021)
  \item Post-prognostics decision-making strategy for a multi-stack fuel cell system (1ème REUNION PLENIERES de la Fédération HYDROGENE (FRH2) du CNRS, Online, 2021)
  \item Post-prognostics decision strategy of a proton exchange membrane fuel cell system (Automatic Control National day, Online, 2020)
\end{itemize}

\section{学术服务}
\subsection{期刊审稿人}
\begin{itemize}
    \item IEEE transactions on transportation electrification (2023-)
    \item Proc.IMechE, Part D: Journal of Automobile SAGE (2022-)
    \item Tungsten Springer (2022-)
    \item Frontiers in Plant Science (2022-)
    \item ISA Transactions Elsevier (2021-)
    \item SAE International Journal of Electrified Vehicles SAE (2021-)
\end{itemize}

%% Reference
%\newpage
%\bibliographystyle{IEEETran}
%\bibliography{mycite}
\end{document}
